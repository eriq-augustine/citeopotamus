% Term paper proposal.

\documentclass[]{IEEEtran}

\usepackage{float}
\usepackage{url}
\usepackage{graphicx}
\usepackage{color}

\begin{document}

\title{AutoCiter: Let's Go Cite-Seeing!}

% author names and affiliations
% use a multiple column layout for up to two different
% affiliations

\author{\IEEEauthorblockN{Eriq Augustine, Aldrin Montana, Ryan Verdon}
\\
\IEEEauthorblockA{Department of Computer Science\\
Cal Poly, San Luis Obispo\\
 \textsf{eaugusti@calpoly.edu, amontana@calpoly.edu, rverdon@calpoly.edu}
}
}

\maketitle

\thispagestyle{empty}
\pagestyle{empty}

\section{Introduction}
We will be doing the Auto-Citer project. We LOVE acedemic papers.

\section{Problem Description}
The probem of auto-citation can be broken up into two cases.

\subsection{Known Location}
In this case, the location of the citation is known, but the proper paper index is unknown.

\subsection{Unknown Location}
In this case, the list of citations is known. However, there is no citation markers in the paper.

\section{Background}
Taken alone, this problem does not solve many questions in the wild. However without loss of generality, these
methods should be able to be used on a larger scale to make citation reccomendations.

\section{Dataset}
We will be acquiring our dataset from the ACM Digital Library. The dataset will include multiple
papers where most of the references are also available on the ACM DL. Both the main paper and the
references will be included in out final dataset.

On the ACM DL, only read-only (usually PDF) versions of the papers are supplied.
To make things easier, we will be pre-prossing the PDFs and converting them into plain text files.

\subsection{Test Dataset}
For our testing dataset, we will remove the citation index from the citation markers. For example,
``[164]'' will become ``[]''. If there are multiple cites in the same sentence, then we will break this up into
many single citations. ``[1, 2, 3]'' or ``[1][2][3]'' will become ``[][][]''. However if we do not have access to the
text of one of the citations, then we will just leave that citation marker. So if citation 2 is not available,
``[1, 2]'' will become ``[][2]''.

For the problem where we do not know where the citation marker belongs, then will just remove all markers from the paper.

\section{Proposed Methods}
We have yet to decide what language we will be using for this project, but we will probably either use the WEKA machine learning
packageweka\cite{weka} or NLTK\cite{nltk}.

\bibliographystyle{acm}
\bibliography{refs}

\end{document}
